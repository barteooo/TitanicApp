\documentclass[12pt,a4paper]{article}
\usepackage[utf8]{inputenc}
\usepackage{dsfont} 
\usepackage[polish]{babel}
\usepackage{amsmath}
\usepackage{graphicx}
\usepackage[top=1in, bottom=1.5in, left=1.25in, right=1.25in]{geometry}

\usepackage{subfig}
\usepackage{multirow}
\usepackage{multicol}
\graphicspath{{Imagens/}}
\usepackage{xcolor,colortbl}
\usepackage{float}

\newcommand \comment[1]{\textbf{\textcolor{red}{#1}}}

%\usepackage{float}
\usepackage{fancyhdr} % Required for custom headers
\usepackage{lastpage} % Required to determine the last page for the footer
\usepackage{extramarks} % Required for headers and footers
\usepackage{indentfirst}
\usepackage{placeins}
\usepackage{scalefnt}
\usepackage{xcolor,listings}
\usepackage{textcomp}
\usepackage{color}
\usepackage{verbatim}
\usepackage{framed}

\definecolor{codegreen}{rgb}{0,0.6,0}
\definecolor{codegray}{rgb}{0.5,0.5,0.5}
\definecolor{codepurple}{HTML}{C42043}
\definecolor{backcolour}{HTML}{F2F2F2}
\definecolor{bookColor}{cmyk}{0,0,0,0.90}  
\color{bookColor}

\lstset{upquote=true}

\lstdefinestyle{mystyle}{
	backgroundcolor=\color{backcolour},   
	commentstyle=\color{codegreen},
	keywordstyle=\color{codepurple},
	numberstyle=\numberstyle,
	stringstyle=\color{codepurple},
	basicstyle=\footnotesize\ttfamily,
	breakatwhitespace=false,
	breaklines=true,
	captionpos=b,
	keepspaces=true,
	numbers=left,
	numbersep=10pt,
	showspaces=false,
	showstringspaces=false,
	showtabs=false,
}
\lstset{style=mystyle}

\newcommand\numberstyle[1]{%
	\footnotesize
	\color{codegray}%
	\ttfamily
	\ifnum#1<10 0\fi#1 |%
}

\definecolor{shadecolor}{HTML}{F2F2F2}

\newenvironment{sqltable}%
{\snugshade\verbatim}%
{\endverbatim\endsnugshade}

% Margins
\addtolength{\footskip}{0cm}
\addtolength{\textwidth}{1.4cm}
\addtolength{\oddsidemargin}{-.7cm}

\addtolength{\textheight}{1.6cm}
%\addtolength{\topmargin}{-2cm}

% paragrafo
\addtolength{\parskip}{.2cm}

% Set up the header and footer
\pagestyle{fancy}
\rhead{\hmwkAuthorName} % Top left header
\lhead{\hmwkClass: \hmwkTitle} % Top center header
\rhead{\firstxmark} % Top right header
\lfoot{Bartłomiej Wnuk} % Bottom left footer
\cfoot{} % Bottom center footer
\rfoot{} % Bottom right footer
\renewcommand{\headrulewidth}{1pt}
\renewcommand{\footrulewidth}{1pt}

    
\newcommand{\hmwkTitle}{Titanic App} % Tytuł projektu
\newcommand{\hmwkDueDate}{\today} % Data 
\newcommand{\hmwkClass}{Technologie chmurowe} % Nazwa przedmiotu
\newcommand{\hmwkAuthorName}{Bartłomiej Wnuk} % Imię i nazwisko

% trabalho 
\begin{document}
% capa
\begin{titlepage}
    \vfill
	\begin{center}
	\hspace*{-1cm}
	\vspace*{0.5cm}
    \includegraphics[scale=0.55]{imagens/loga.png}\\
	\textbf{Uniwersytet Gdański \\ [0.05cm]Wydział Matematyki, Fizyki i Informatyki \\ [0.05cm] Instytut Informatyki}

	\vspace{0.6cm}
	\vspace{4cm}
	{\huge \textbf{\hmwkTitle}}\vspace{8mm}
	
	{\large \textbf{\hmwkAuthorName}}\\[3cm]
	
		\hspace{.45\textwidth} %posiciona a minipage
	   \begin{minipage}{.5\textwidth}
	   Projekt z przedmiotu technologie chmurowe na kierunku informatyka profil praktyczny na Uniwersytecie Gdańskim.\\[0.1cm]
	  \end{minipage}
	  \vfill
	%\vspace{2cm}
	
	\textbf{Gdańsk}
	
	\textbf{\hmwkDueDate}
	\end{center}
	
\end{titlepage}

\newpage
\setcounter{secnumdepth}{5}
\tableofcontents
\newpage

\section{Opis projektu}
\label{sec:Project}
Australijski miliarder postanowił odbudować Titanica – nowy statek ma być oddany do użytku w 2027 roku. To właśnie wtedy będzie można zapisać się na rejs i przepłynąć trasę, która była zaplanowana na 1912 rok. Firma, która będzie zajmować się organizacją rejsu poprosiła o stworzenie aplikacji, która będzie pozwalała na zapisanie się na rejs oraz sprawdzenie szans na przeżycie gdyby okazało się, że statek znowu uderzy w górę lodową i zacznie tonąć.

\subsection{Opis architektury - 8 pkt}
\label{sec:introduction}
Architektura systemowa projektu opiera się na podejściu mikroserwisów, co oznacza, że różne funkcjonalności aplikacji są realizowane przez niezależne komponenty (mikroserwisy), które komunikują się ze sobą. Aplikacja jest zbudowana z pięciu serwisów hostowanych na klastrze Kubernetes \\
    \\
    1. Serwis frontendowy - interfejs użytkownika \\
    2. Baza danych \\
    3. Serwis backendowy - zarządzanie bazą danych\\
    4. Serwis integracji z klientem uczenia maszynowego \\
    5. Serwis autentykacji Keycloak \\

\subsection{Opis infrastruktury - 6 pkt}
\label{sec:Users}

\subsubsection{Docker}
Każdy mikroserwis jest
zapakowany w osobny kontener Dockerowy. Kontenery zawierają wszystkie
zależności potrzebne do uruchomienia aplikacji.
\subsubsection{Docker-compose} 
Służy do definiowania i uruchamiania wielokontenerowych
aplikacji Dockerowych. Plik docker-compose.yml opisuje konfigurację
kontenerów, sieci i wolumenów dla lokalnego środowiska deweloperskiego.
\subsubsection{Kubernetes} 
Aplikacja jest zaprojektowana do działania w środowisku Kubernetes, wykorzystując Docker Desktop jako platformę deweloperską. Dzięki temu środowisku możliwa jest szybka orkiestracja kontenerów Docker, co jest idealne dla celów deweloperskich i testowych. Do zarządzania klastrami Kubernetes używamy narzędzia `kubectl`, co umożliwia efektywne wdrażanie, skalowanie oraz monitorowanie usług. Konfiguracja w Docker Desktop zapewnia izolowane sieci(każdy kontener ma swoją własną odseparowaną sieć) i zarządzanie pamięcią masową za pomocą Persistent Volumes. Pozwalają one na przechowywanie danych bez względu na cykl życia poszczególnych podów. 

\subsection{Opis komponentów aplikacji - 8 pkt}
\label{sec:FunctionalConditions} \\
\subsubsection{Serwis frontendowy - interfejs użytkownika}
Stworzony przy użyciu technologi React. Umożliwia logowanie istniejących użytkowników. Zawiera również w sobie interfejs odpowiadający za wyświetlenie predykcji przeżycia. Konto administartora zawiera interfejs umożliwiający usuwanie użytkowników z bazy danych.
\subsubsection{Baza danych} 
Stworzona przy użyciu Postgresql. Zawiera osoby jakie wzięły udział w wyprawie w 1912 roku oraz jest przygotowana na dodawanie nowych użytkowników, którzy chcieliby zapisać się na rejs. 
\subsubsection{Serwis backendowy - zarządzanie bazą danych} 
Stworzony przy użyciu technologi FastAPI w Pythonie. Umożliwia wykonywanie operacji na członkach załogi przez integrację z bazą danych - ich tworzenie, usuwanie oraz odczytywanie.
\subsubsection{Serwis integracji z klientem uczenia maszynowego} 
Stworzone przy użyciu technologi FastAPI oraz Scikit Learn w pythonie. Zawiera wytrenowany model oraz API do odbierania danych wejściowych, które są przekazywane do modelu w celu zwrócenia wyniku predykcji. 
\subsubsection {Serwis autentykacji Keycloak}  
Został użyty do zabezpieczenia frontendu i umożliwienia bezpiecznego logowania z uwzględnieniem rol użytkowników.
 
\subsection{Konfiguracja i zarządzanie - 4 pkt}
\label{sec:NonFunctionalConditions} 
\subsubsection {Deploymenty} 
Każdy komponent aplikacji jest uruchamiany jako osobny Deployment w Kubernetes. Deploymenty pozwalają na łatwe skalowanie i aktualizację komponentów aplikacji.
\subsubsection {ConfigMaps} 
Zarządzanie w systemie odbywa się za pomocą ConfigMap w Kubernetesie, co umożliwia wstrzykiwanie wartości per środowisko.
\subsubsection {Persistent Volume Claims} 
Dane bazy danych są przechowywane na Persistent Volume za pomocą Persistent Volume Claim - usługi, która jest żądaniem o przydzielenie zasobu przechowywania danych określonego w Persistent Volume.

\subsection{Zarządzanie błędami - 2 pkt}
\label{sec:ERD} 
\subsubsection{Aplikacja}
Zarządzanie błędami na poziomie aplikacji skupia się na walidacji requestów HTTP, co pozwala na wczesne wykrycie i obsługę błędnych danych. 
\subsubsection{Kubernetes} 
Dla deploymentu każda usługa korzysta z mechanizmu replikacji, który przekieruje do drugiego poda gdy pierwszy napotka problem.

\subsection{Skalowalność - 4 pkt}
\label{sec:ExamplesSection}
Skalowanie horyzontalne - pozwala na automatyczne skalowanie liczby replik kontenerów w zależności od obciążenia.Liczba replik w deploymentach serwisów została zwiększona do dwóch co pozwala na lepsze rozłożenie obciążenia dla każdego z komponentów aplikacji.

\subsection{Wymagania dotyczące zasobów - 2 pkt}
\label{sec:ExampleTables}




\subsection{Architektura sieciowa - 4 pkt}
\label{sec:ExampleResults}
\subsubsection{Protokół http} 
 - komunikacja frontendu z keycloakiem \\
 - komunikacja frontendu z backendem 
 \subsubsection{Protokół bazodanowy postgresql}
 - komunikacja backendu z bazą danych 
  \subsubsection{Protokół OAuth2 z keycloak}
- zarządzanie sesją użytkownika oraz jej autoryzacją
\noindent


\bibliographystyle{amsplain}
\bibliography{references.bib}
\nocite{*}

\end{document}z